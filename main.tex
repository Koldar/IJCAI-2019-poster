
\documentclass[%
    25pt, a1paper, portrait,%
]{tikzposter}

%%%%%%%%%%%%%%%%%%%%%%%%%%%%%%%%%%%%%%%%%%%%
% PACKAGES
%%%%%%%%%%%%%%%%%%%%%%%%%%%%%%%%%%%%%%%%%%%%

\usepackage[english]{babel}
\usepackage{amsmath, amsthm, amssymb, latexsym}
\usepackage[utf8]{inputenc}
\usepackage{etoolbox}
\usepackage{xparse}
\usepackage{subcaption}
\usepackage{tikz}
\usepackage{blindtext}
\usepackage{geometry}
\usepackage{calculator}


\NewDocumentCommand{\code}{m}{%
    \texttt{#1}%
}

\NewDocumentCommand{\CPD}{}{%
    \textsf{CPD}%
}

\NewDocumentCommand{\setFontSize}{m o m}{%
    \IfNoValueF{#2}{%
        \fontsize{#1}{#2}\selectfont#3%
    }{%
        % see https://texblog.org/2012/08/29/changing-the-font-size-in-latex/
        \MULTIPLY{#1}{1.2}{\setFont@baseline}%
        \fontsize{#1}{\setFont@baseline}\selectfont#3%
    }%
}

%%%%%%%%%%%%%%%%%%%%%%%%%%%%%%%%%%%%%%%%%%%%
% TITLE
%%%%%%%%%%%%%%%%%%%%%%%%%%%%%%%%%%%%%%%%%%%%

\title{Path Planning with CPD Heuristics}
\author{Massimo Bono$^1$, Alfonso E. Gerevini$^1$, Daniel D. Harabor$^2$, Peter J. Stuckey$^2$}
\date{\today}
\institute{%
    $^1$\setFontSize{42}{Dipartimento Di Ingegneria dell'Informazione, Università degli Studi di Brescia, Brescia, Italy}%
    \\%
    $^2$\setFontSize{42}{Faculty of Information Technology, Monash University, Melbourne, Australia}%
}

%%%%%%%%%%%%%%%%%%%%%%%%%%%%%%%%%%%%%%%%%%
% STYLE
%%%%%%%%%%%%%%%%%%%%%%%%%%%%%%%%%%%%%%%%%%

\usetheme{Autumn}
\usecolorstyle[colorPalette=BrownBlueOrange]{Germany}

%%%%%%%%%%%%%%%%%%%%%%%%%%%%%%%%%%%%%%%%%%
% configuration of tikzposter
%%%%%%%%%%%%%%%%%%%%%%%%%%%%%%%%%%%%%%%%%%

%remove logo https://tex.stackexchange.com/a/263278/145331
\tikzposterlatexaffectionproofoff

%%%%%%%%%%%%%%%%%%%%%%%%%%%%%%%%%%%%%%%%%%%%
% DOCUMENT
%%%%%%%%%%%%%%%%%%%%%%%%%%%%%%%%%%%%%%%%%%%%

\begin{document}

    \maketitle
    
    \block{Abstract}{%
        Compressed Path Databases (CPDs) are a leading technique for optimal pathfinding in graphs with static edge costs. In this work we investigate CPDs as admissible heuristic functions and we apply them in two distinct settings: problems where the graph is subject to dynamically changing costs, and anytime settings where deliberation time is limited. Conventional heuristics derive cost-to- go estimates by reasoning about a tentative and usually infeasible path, from the current node to the target. CPD-based heuristics derive cost-to-go estimates by computing a concrete and usually feasible path. We exploit such paths to bound the optimal solution, not just from below but also from above. We demonstrate the benefit of this approach in a range of experiments on standard gridmaps and in comparison to Landmarks, a popular alternative also developed for searching in explicit state-spaces.
    }

    \begin{columns}
        \column{0.3}
            \block{Title block}{%
                Body block
            }
        
        \column{0.7}
            \block{Title block}{%
                Body block
            }    
    \end{columns}
    
\end{document}